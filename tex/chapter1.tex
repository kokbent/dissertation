\chapter{INTRODUCTION} \label{introduction}

Plant phenology, the timing of recurring biological events such as flowering, plays an important role in ecological research extending from local to global scales \citep{cleland2007, richardson2013, tang2016}. At large scales the timing of spring leaf out and fall senescence influence the carbon budget of earth system models, which has implications for correctly accounting for biosphere-atmosphere feed backs in long-term climate forecasts \citep{richardson2012}. At smaller scales, species-specific responses to temperature and precipitation can alter flower communities \citep{diez2012, caradonna2014, theobald2017} and affect the abundance and richness of both pollinators \citep{ogilvie2017a, ogilvie2017b} and organisms at higher trophic levels \citep{tylianakis2008}. 

Because of the central importance of phenology, advanced forecasts for when phenological events will occur have numerous potential applications including: 1) research on the cascading effects of changing plant phenology on other organisms; 2) tourism planning related to flower blooms and autumn colors; 3) planning for sampling and application of management interventions by researchers and managers; and 4) agricultural decisions on timing for planting, harvesting, and application of pest prevention techniques. However, due to the challenges of automatically integrating, predicting, and disseminating large volumes of data, there are limited examples of applied phenology forecast systems. Plant phenology models that are robust at multiple ecological scales, or deemed appropriate for a particular scale, are needed to better understand and forecast the timing of key biological events. Additionally, phenology data at a suitable spatial extent has only recently come online and poses several challenges for validation and integration into phenological models. 

Plant phenology models have historically been parameterized using data from long-term, well-sampled study sites. Yet data from a single location might not be suitable for large-scale phenology forecasts. Large-scale data is available, but comes from citizen science programs where several potentially biases can be introduced \citep{dickinson2010}. A thorough evaluation of the data capability is needed before integrating citizen science data into the pipeline for a plant phenology forecast. Data validation will ensure that any potential biases in the phenological observations are either non-existent, or too small to affect phenological models. 

The variable of interest in plant phenology models is the transition date. For example the date a flower blooms or leaf emerges is the onset transition date. Other common transition dates are first observed open flowers or new leafs on a plant, but can also include peak flower, fruit maturation, and leaf senescence. Historic datasets often use repeated observations to identify the true transition date  \citep{davis2015, wolkovich2012}, yet this is susceptible to observer bias  \citep{miller-rushing2008}. Most modern studies and collection protocols use status-based monitoring, where observers record the current state of a plant (ie. leaves present or absent) without necessarily conducting repeated observations to observe recent or future transitions. This includes research using herbarium records, where the presence or absence of flowers and other phenophases is inferred from their presence on a specimen  \citep{willis2017}. To make use of status-based data in most phenological models the transition date must first be estimated, and to date few comprehensive comparison of these estimators have been performed. Here I perform an evaluation of several transition date estimators using an 11 year observational dataset. 

Large-scale forecast systems are are complex pipelines which ingest, process, and combine several continuous streams of data. The knowledge acquired and lessons learned in building these systems will help define best practices and needed tools and training for other groups to implement similar systems \citep{yenni2019}. A thorough description of the pipeline, and the challenges which were overcome in building it, will help advance the field of ecological forecasting. This pipeline description, along with validation of phenological data, are the first steps in implementing a continuously updated phenology forecast system. 


