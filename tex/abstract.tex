% Write in only the text of your abstract, all the extra heading jargon is automatically taken care of
\begin{abstract}
Despite phenology being an integral part of ecological systems there is currently no near-term plant phenology forecast due to several challenges. Phenological data at an adequate scale in North America is from volunteer citizen science programs, and due to potentially observer bias may not be suitable for driving predictive models. These data are also status-based, in that they do not indicate transition dates directly, such as the timing of flowering and leaf out, but instead indicate the presence or absence of these phenophases. Implementing a forecast system also has numerous computational challenges, such as managing models for numerous species, ingesting large-scale climate forecasts, and disseminating those forecasts in a reliable and interpretable way. This dissertation addresses three questions. First we compared citizen science phenological data with data from long-term research studies. We found that models built with these different data sets resulted in different parameters, potentially resulting in conflicting inferences. Yet models from the two data sources produced similar estimates for phenological events, suggesting the large-scale citizen science data are suitable for predictive modeling. Second, we used a 10 year dataset of near daily phenological observations to test different methods of estimating transition dates. The best estimator depended on the scale of the data, with a method derived from a Weibull distribution being the best for a population of plants, while a method using the first observed date of a phenophase, combined with the most recent absence, was the best for individual plants. Finally, we implemented an automated plant phenology forecast system which predicts the timing of budburst, flowers, ripe fruit, and fall colors for 77 species across the United States up to 6 months in advance. We describe the key steps in the pipeline, including: 1) fitting phenology models, 2) acquiring climate data; 3) making predictions for phenological events; 4) disseminating those predictions; and 5) fully automating the pipeline. From the lessons learned we provide guidance in developing large-scale near-term ecological forecast systems more generally and to help advance the use of automated forecasting in ecology. 
\end{abstract}
