\chapter{CONCLUSION} \label{conclusion}

For the validation of citizen science source phenologcial data our results suggest that both LTER and USA-NPN data provide valuable information on plant phenology. Models built using both data sources yield effective predictions for phenological events, but parameter estimates from the two data sources differ and models from each source best predict that data source's phenology events. The primary difference in the datasets is spatial scale, but due to trade-offs in data collection efforts, the larger scale USA-NPN data have shorter time-series, less site fidelity and other differences from the intensively collected LTER data (Table \ref{table-2-3}). These differences can be strengths or potential limitations. Observers sampling opportunistically allows the USA-NPN dataset to have a large spatial scale, but also leads to low site fidelity, which limits the ability to measure long-term trends at local scales \citep{gerst2016}. Tracking long-term trends is the major strength of LTER data, but having a relatively small species pool limits their use in species-level predictive modeling. Due to these differences, the best data source for making predictions depends on the scale at which the predictions are being made. Identifying the most effective data sources for different types and scales of analysis is a useful first step, but the ultimate solution to working with diverse data types is to focus on integrating all types of data into analyses and forecasts \citep{hanks2018, melaas2016}. Our results suggest that methods that can learn from the intensive information available in LTER data in regions where they are available, and simultaneously use large-scale data to capture spatial variation in phenological requirements will help improve our ability to understand and predict phenology. 

In validating the phenological estimators I have used a precise flowering dataset to confirm that naively using the first flowering observation is biased, and estimates using the mean flowering reliable for estimating flowering peak. I have also shown how the recently introduced Weibull method can produce reliable estimates given an adequate sample size. The Logistic and GAM methods can be useful with large datasets having low amounts of flowering presence, and future collection efforts should emphasize absence observations for this reason. Additionally, estimating transition dates of individual plants is best done with the Midway method using a 7 day restriction, and the Weibull method if the restriction results in a low number of final samples. These estimators are needed for translating status-based phenological data into distinct transition dates used to track and forecast changing seasonal patterns.

In conclusion, using recent advances in open source software and large-scale open data collection we have implemented an automated high resolution, continental scale, species-level phenology forecast system. Implementing a system of this scale was made possible by a new phenology data stream and new computational tools that facilitate large scale analysis with limited computing and human resources. Most recent research papers describing ecological forecast systems focus on only the modelling aspect \citep{chen2011, carrillo2018, vandoren2018}, and studies outlining implementation methods and best practices are lacking (but see \cite{white2018, welch2019}). Making a forecast system operational is key to producing applied tools, and requires a significant investment in time and other resources for data logistics and pipeline development. Major challenges here included the automated processing of large meteorological datasets, efficient application of hundreds of phenological models, and stable, consistently updated, and easy to understand dissemination of forecasts. By discussing our approach to automated forecasting on data-intensive questions, and making our code publicly available, we hope to provide guidance for others developing ecological forecasting systems. While many areas for improvement remain for this system, including improved phenology models and more user friendly dissemination of forecasts, this system provides a fully automated actionable forecasts on large scale phenology.